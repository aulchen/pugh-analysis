\documentclass{article}
\usepackage{amsmath}
\usepackage{amsfonts}
\usepackage{amssymb}
\usepackage{bbm}
\usepackage{enumitem}


\newenvironment{proof}{\paragraph{Proof:}}{\hfill$\square$}
\newtheorem{theorem}{Theorem}
\newtheorem{lemma}[theorem]{Lemma}
\newtheorem{corollary}[theorem]{Corollary}

\newcommand{\R}{\mathbb{R}}
\newcommand{\Q}{\mathbb{Q}}
\newcommand{\Z}{\mathbb{Z}}
\newcommand{\N}{\mathbb{N}}

\author{Arthur Chen}
\title{Chapter 5 Multivariable Calculus}
\date{\today}

\begin{document}
\maketitle

\section*{Problem 1}

Let $T: V \rightarrow W$ be a linear transformation, and let $p \in V$ be given. Prove that the following are equivalent:

\begin{enumerate}[label=(\alph*)]
\item $T$ is continuous at the origin.
\item $T$ is continuous at $p$.
\item $T$ is continuous at at least one point of $V$.
\end{enumerate}

For (a) to (b), let $p_n \in V$ be an arbitrary sequence such that $p_n \rightarrow p$. I claim that this implies $T(p_n) \rightarrow T(p)$ in $W$, meaning that $T$ is continuous at $p$.

Although we can not use the translation invariance of the norm, we can derive a similar property for sequences when $X$ is a vector space.

\begin{lemma}
Let $V$ be a vector space. Translation is a continuous function.
\begin{proof}
Let $T: V \rightarrow V$ be defined such that $T(x) = x + y$, $x, y \in V$ are arbitrary, $y$ is fixed. Let $\epsilon > 0$ be arbitrary, and let $\delta = \frac{\epsilon}{2}$. Then the translation of the $\delta$-ball around $x$ to the $\delta$-ball around $x+y$ lies inside the $\epsilon$-ball around $x+y$.

Specifically, $B_(x, \delta)$ is defined as

\textbf{TO FINISH}

\end{proof}
\end{lemma}


\begin{lemma}
Let $V$ be a vector space and $d$ be a metric. Let $p_n \rightarrow p$ and $x \in X$ be arbitrary. Then $p_n + x \rightarrow p+x$.
\begin{proof}
Translation is continuous, and the product of continuous functions is continuous. All metrics are continuous, and the composition of continuous functions is continuous. Thus $p_n \rightarrow p$ implies

\[
\lim_{n \rightarrow \infty} d(p_n + x, p + x) = d(p + x, p + x) = 0
\]

by continuity.
\end{proof}
\end{lemma}

Letting $d_W$ be the metric on $W$, we have

\[
d_W(T(x_n + p), T(p)) = d_W(T(x_n) + T(p)
\]

\end{document}